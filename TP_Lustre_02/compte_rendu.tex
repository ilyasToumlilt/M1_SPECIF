\documentclass[a4paper, 11pt]{article}
 
\usepackage[utf8]{inputenc}
\usepackage{graphicx}
\usepackage[frenchb]{babel}
\usepackage{tikz}
\usetikzlibrary{arrows,automata}
\usepackage{makecell}

\begin{document}
 
\title{SPECIF - Lustre\\Compte-rendu TP2}
\author{Ilyas Toumlilt}
\date{02/03/2015}
 
\maketitle
 
\paragraph{Question 1}
Voir: src/metronome.lus

\paragraph{Question 2}
Les variables présentes dans le code C généré correspondent aux variable Lustre suivantes : 
\begin{itemize}
  \item \_V1 : reset
  \item \_V2 : delay
  \item \_V3 : true$\rightarrow$false
  \item \_V4 : hz
  \item \_V5 : first
  \item \_V6 : n
  \item \_V7 : state
  \item \_V8 : tic
    \item \_V9 : tac
\end{itemize}

\paragraph{}
Soit $D = \{first, hz, n, state, tic, tac\}$, avec $first$ initialisé à \texttt{false}.

\begin{tikzpicture}[->,>=stealth',shorten >=1pt,auto,node distance=2.8cm,
                    semithick]
  \tikzstyle{every state}=[fill=white,text=black]

  \node[initial,state] (s0) {$s_0$};
  \path (s0) edge [loop right] node 
        {\makecell[l]{[false]\\
            lire $reset, delay$ \\
            first $\leftarrow$ reset \textbf{or} false$\rightarrow$pre first\\
            hz $\leftarrow$ \textbf{if} first and reset \textbf{then} delay \textbf{else} 0$\rightarrow$pre hz\\
            n $\leftarrow$ \textbf{if} hz = 0 \textbf{then} 0 \textbf{else} 0$\rightarrow$(pre(n)-1) mod hz\\
            state $\leftarrow$ \textbf{if} n = 0 \textbf{then} false$\rightarrow$(not pre(state)) \textbf false$\rightarrow$pre(state)\\
            tic $\leftarrow$ ( not reset ) and first and ( n = 0 ) and state\\
            tac $\leftarrow$ ( not reset ) and first and ( n = 0 ) and not state
        }} (s0);
\end{tikzpicture}

\paragraph{Question 3}
Les valeurs lustres correspondant aux variables du code C généré par la commande lus2oc -2 sont les suivantes :
\begin{itemize}
  \item \_V1 : reset
  \item \_V2 : delay
  \item \_V3 : hz > 0
  \item \_V4 : hz
  \item \_V5 : n==0
  \item \_V6 : n
  \item \_V7 : n > 0
  \item \_V8 : tic
  \item \_V9 : tac
\end{itemize}

\end{document}
